\documentclass{article}
\usepackage[utf8]{inputenc}
\usepackage[T1]{fontenc}
\usepackage[german]{babel}
\usepackage{listings}
\usepackage[a4paper]{geometry}
\usepackage{hyperref}

\title{IBRC}
\author{Gruppe 5}
\date{\today}

\begin{document}

\maketitle

\section{Protokoll}

Der Server direkt mit dem Client verbundene Server prüft immer, ob source mit dem Absender übereinstimmt. Besser als nichts. 

Die Pakete sehen so aus:
\begin{lstlisting}
----------------------- - -
|name|param1|param2|...
----------------------- - -
\end{lstlisting}
wobei die Felder jeweils durch Leerzeichen getrennt sind und name der Name des Paketes ist.

\subsection{NICK}

\begin{enumerate}
  \item Der Client sendet an den Server NICK = (source: NICK, nick: NICK).
  \item
    \begin{enumerate}
      \item Falls source nicht in der CLIENTLIST dem absendenden Teilbaum zugeordnet ist, sende STATUS = (dest = source, status = 301) an den Client und brich ab.
      \item Falls nick schon vergeben ist, sende STATUS = (dest = source, status = 302) an den Client und brich ab.
      \item Falls der Wurzelknoten erreicht wurde, und kein Fehler aufgetreten ist, sende STATUS = (dest = source, status = 300) an den Client und NICKRES = (dest = source, nick = nick).
    \end{enumerate}
  \item Wiederhole den vorherigen Punkt. 
\end{enumerate}

\subsection{JOIN}

\begin{enumerate}
  \item Der Client sendet JOIN = (source: NICK, channel: CHANNEL.name) an den Server.
  \item Falls der Client keinen Nick hat (hat noch kein NICK gesendet), sendet der Server STATUS mit Fehlercoe 303 an den Client.
  \item Falls der Client bereits einen CHANNEL hat, sendet der Server STATUS = (dest = source, status = 404) und bricht ab. 
  \item
    \begin{enumerate}
      \item Der Server prüft in seiner CHANNELLIST, ob er channel kennt.
      \item Der Server trägt den Absender von JOIN (Client oder ein anderer nächster Hop) in die CHANNELLIST ein.
      \item Falls der Channel schon bekannt war, sende STATUS = (dest = source, status = 400) und CHANNEL = (dest = source, channel: CHANNEL) an source und breche ab.
      \item Falls der Channel noch nicht bekannt war, sende JOIN an den Elternknoten weiter.
      \item Falls kein Elternknoten mehr existiert, sende STATUS = (dest = source, status = 401) an den Client und sende CHANNEL = (dest = source, channel: CHANNEL) an den Client und breche ab.
    \end{enumerate}
  \item Alle Server, die CHANNEL erhalten, übernehmen den Eintrag in ihre CHANNELLIST.
\end{enumerate}

\subsection{LEAVE}

\begin{enumerate}
  \item Der Client sendet LEAVE = (source: NICK, channel: CHANNEL.name) an den Server.
  \item
    \begin{enumerate}
      \item Der Server prüft, ob der Absender in channel ist. Falls nicht sendet er STATUS = (dest = source, STATUS = 402) an den letzen Absender.
      \item Der Server löscht die NICK des letzen Absenders aus der CHANNELLIST für channel.
      \item Falls die Liste für channel keine Einträge mehr enthält, wird der Eintrag gelöscht und falls zudem ein Elternknoten existiert, sende ein LEAVE an den Elternknoten. Wiederholen.
    \end{enumerate}
  \item Falls kein weiterer Elternknoten mehr existiert, brich ab.
\end{enumerate}

\subsection{LIST}

\begin{enumerate}
  \item Der Client sendet LIST = (source: NICK) an den Server.
  \item Die Server leiten LIST an ihre Wurzel.
  \item Der Server, welcher alle Kanäle kennt (Wurzel), antwortet mit LISTRES = (dest = source, channels = [CHANNEL,\ldots]).
\end{enumerate}

\subsection{GETTOPIC}

\begin{enumerate}
  \item Der Client sendet GETTOPIC = (source: NICK, channel: CHANNEL.name) an den Server.
  \item Der erste Server, der den channel kennt, antwortet mit TOPIC = (dest = source, channelname: CHANNEL.name, topic: CHANNEL.topic).
  \item Falls kein Server channel kennt, antwortet die Wurzel mit STATUS = (dest = source, status = 403).
\end{enumerate}

\subsection{SETTOPIC}

\begin{enumerate}
  \item Der Client sendet SETTOPIC = (source: NICK, channel: CHANNEL.name, topic: CHANNEL.topic) an seinen Server.
  \item
    \begin{enumerate}
      \item Ist CHANNEL.op nicht gleich source, dann sendet der Server STATUS = (dest = source, status = 502) an den Client.
      \item Ansonsten übernimmt der Server topic in seine CHANNELLIST und sendet SETTOPIC weiter an den Kanal (siehe Routing)
    \end{enumerate}
\end{enumerate}

\subsection{MSG}

\begin{enumerate}
  \item Der Client sendet MSG = (source: NICK, channel: CHANNEL.name, msg: STRING) an seinen Server.
  \item Kennt er den Channel nicht, sendet der Server STATUS = (source, status = 403).
  \item Ist der Client nicht in diesem channel angemeldet, sendet der Server STATUS = (dest = source, status = 402).
  \item Der Server sendet MSG an channel (siehe Routing).
  \item Der Server sendet bei Erfolg (sofort) STATUS = (source, status = 600) an den Client.
\end{enumerate}

\subsection{PRIVMSG}

\begin{enumerate}
  \item Der Client sendet PRIVMSG = (source: NICK, channel: CHANNEL.name, dest: NICK, msg: STRING) an dest (siehe Routing)
  \item Kennt ein Server channel nicht, antwortet er an source mit STATUS = (source, status = 403).
  \item Kennt ein Server channel und erhält der die Nachricht von seinem Elternknoten und kennt kennt dest nicht, sendet er STATUS = (source, status = 601) an source.
  \item Kennt ein Server channel und hat er keinen Elternknoten, kennt aber dest nicht, so antwortet er mit STATUS = (source, status = 601)
  \item Kennt ein Server sowohl channel, als auch dest, aber wiedersprechen sich die NICK zu beiden, sendet der Server STATUS = (source, status = 603).
  \item Wurde die Nachricht erfolgreich zugestellt quittiert der Empfänger (Client) die Nachricht mit STATUS = (source, status = 604)
\end{enumerate}

\subsection{QUIT}

\begin{enumerate}
  \item Der Client sendet QUIT an seinen Server. 
  \item Dieser löscht den Client aus allen Kanälen 
  \item und sendet QUIT an seinen Elternknoten weiter.
\end{enumerate}


\section{Datenstrukturen}

\subsection{NICK}

NICK = (nick: STRING[a-zA-Z0-9])\\
Der Nick kann beliebig gewechselt werden, muss aber im Netzwerk einzigartig sein. (siehe NICK).

\subsection{CLIENT}

CLIENT = (nick: NICK, route: int)\\
Jeder CLIENT wird über seinen Nick und seine NICK identifiziert. Die route ist die Route zum nächsten Hop auf dem Weg zum Ziel.

\subsection{CHANNEL}

CHANNEL = (name: STRING, topic: STRING, op: NICK)\\
name wird nach dem ersten JOIN nicht mehr geändert (siehe JOIN). topic wird durch SETTOPIC geändert. TOPIC und CHANNEL aktualisieren die Daten im Netz, wenn Änderungen auftreten.

\subsection{CLIENTLIST}

CLIENTLIST = [(NICK, CLIENT), \ldots]
Speichert die Routinginformationen für die Clients. Die NICK ist hierbei die NICK des nächsten bekannten hops auf dem Pfad zum Client. Hierbei hat jeder Server nur Einträge für Clients in seinen Kindern. Die NICK ist eindeutig, weil ein Client nur in einem Teilbaum sein kann. Kommen Clients hinzu (CONNECT) oder verlassen (DISCONNECT) sie das Netz wird die Liste aktualisiert.

\subsection{CHANNELLIST}

CHANNELLIST = [(CHANNEL, [route, \ldots]), \ldots]
Speichert die Routinginformationen für die Kanäle. Hierbei hat jeder Server nur Einträge für Kanäle in denen Clients seiner Kinder sind. Zu jedem Kanal ist eine Liste von NICK gespeichert, weil in mehreren Teilbäumen Clients in dem Kanal existieren können. Die route sind die Routen der nächsten hops, die den Kanal kennen, also Nachrichten an den Kanal erhalten müssen. Die Liste wird bei JOIN und LEAVE aktualisiert.


\section{Verbindungen}

\begin{lstlisting}
        S
      / | \
    S   C  S
  / | \    |
C   C  C   C
\end{lstlisting}

Server verbinden sich zu je genau einem Elternknoten (anderer Server). Über die gesamte Lebensdauer des Netzes bleibt diese Verbindung bestehen.
Jeder Client ist genau zu einem Server verbunden.
Clients können dynamisch Verbindungen zu Servern abbauen und aufbauen (Siehe NICK, DISCONNECT)


\section{Routing}

\subsection{one-to-one}

\begin{lstlisting}
       S
     /   \
  C-S    S-S-C
\end{lstlisting}

Private Nachrichten sind an den NICK eines Clients in einem CHANNEL addressiert. Kennt ein Server den NICK nicht, so befindet sich der Client auch nicht in seinem Teilbaum. In diesem Fall sendet der Server die Nachricht an seinen Elternknoten. Kennt kein Server die Nachricht, existiert nach dem Protokoll von JOIN kein solcher Client. Der Elternknoten, der sowohl CHANNEL, als auch CLIENT kennt, stellt die Nachricht an den nächsten Kinderknoten, der für CLIENT in seiner CLIENTLIST steht.

\subsection{one-to-many}

\begin{lstlisting}
      S
    / | \
   S  C  S-C
 / | \
C  C  S-C
\end{lstlisting}

Kanalnachrichten werden mit dem CHANNEL.name addressiert. Solche Nachrichten werden an den Elternknoten und alle Einträge in der NICK-Liste zum Kanal weitergeleitet. Ausgenommen von der Weiterleitung ist jedesmal der NICK, von der die Nachricht unmittelbar erhalten wurde.


\section{Fehlerbehandlung}

Siehe Protokoll.

\subsection{Fehler-Codes}

\begin{tabular}{rl}
  100 & CONNECT erfolgreich \\
  101 & Fehler bei CONNECT \\
  200 & DISCONNECT erfolgreich \\
  201 & Fehler bei DISCONNECT \\
  300 & NICK ist einzigartig \\
  301 & Nicht authotisierte NICK-Änderung \\
  302 & NICK ist nicht einzigartig \\
  303 & NICK ist nicht gesetzt worden \\
  400 & Bekannter Channel, JOIN erfolgreich \\
  401 & Neuer Channel erstellt, JOIN erfolgreich \\
  402 & Nicht in Channel \\
  403 & Kein solcher Channel existiert \\
  404 & Bereits in anderem CHANNEL beigetreten \\
  405 & Kanal erfolgreich verlassen \\
  502 & Absender ist nicht Channel-OP \\
  600 & Kanalnachricht erfolgreich zugestellt \\
  601 & Kein solcher Client im Channel \\
  603 & Client befindet sich woanders, aber nicht in dem Kanal \\
  604 & Private Nachricht erfolgreich zugestellt \\
\end{tabular}

\subsection{Gleichzeitigkeit}

\emph{Situation}: TOPIC ändert sich. CHANNEL paket noch nicht bei allen Servern eingetroffen.\\
\emph{Lösung}: Vernachlässigbar, da der Client beim nächsten GETTOPIC das aktuelle TOPIC erhalten wird. Verursacht keine wiedersprüchlichen Daten im Sinne des Protokoll.\\

\emph{Situation}: Es wird von zwei Clients über verschiedene Server jeweils ein JOIN gesendet. Der Kanal wurde noch  nicht erstellt.\\
\emph{Lösung}: Der Wurzelknoten erhält genau eins der beiden JOIN zuerst. Dieses wird genutzt, um den Kanal zu erstellen und der Absendende Client wird Kanal-Op. Der Server sendet STATUS mit Code 401 an diesen Client. Der andere Client erhält STATUS mit Code 400. An beide Clients wird CHANNEL vom Wurzelserver gesendet. Dadurch erhalten alle Server auf dem Pfad zum jeweiligen CLient die Kanalinformationen und überschreiben die beim JOIN erstellten, vorläufigen, Information mit den aktuellen von der höheren Hirarchiebene erstellten Kanalinformationen.

\end{document}
